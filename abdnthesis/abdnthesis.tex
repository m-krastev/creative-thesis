\documentclass[bsc]{abdnthesis}

%% For citations, I would recommend natbib for its                          
%% flexibility, particularly when named citation styles are used, but                
%% it also has useful features for plain and those of that ilk.                      
%% The natbib package gives you the following definitons                             
%% that extend the simple \cite:                                                     
%   \citet{key} ==>>                Jones et al. (1990)                              
%   \citet*{key} ==>>               Jones, Baker, and Smith (1990)                   
%   \citep{key} ==>>                (Jones et al., 1990)                             
%   \citep*{key} ==>>               (Jones, Baker, and Smith, 1990)                  
%   \citep[chap. 2]{key} ==>>       (Jones et al., 1990, chap. 2)                    
%   \citep[e.g.][]{key} ==>>        (e.g. Jones et al., 1990)                        
%   \citep[e.g.][p. 32]{key} ==>>   (e.g. Jones et al., p. 32)                       
%   \citeauthor{key} ==>>           Jones et al.                                     
%   \citeauthor*{key} ==>>          Jones, Baker, and Smith                          
%   \citeyear{key} ==>>             1990                                             

\usepackage[round,colon,authoryear]{natbib}
\usepackage[acronym]{glossaries}
\setlength{\bibsep}{0pt}
\bibliographystyle{apalike}
\usepackage{hyperref}
\usepackage{booktabs}
\usepackage{graphicx}
\usepackage{subcaption}
\usepackage{multirow}
\usepackage{listings}

\usepackage[T1]{fontenc}

\title{Towards Evaluating Creativity in Language}
\author{Matey Krastev}
% IMO this is a bit silly, but some like to include these. To remove,
% delete this declaration and remove the option from the
% \documentclass definition above.
%\qualifications{PhD, Computer Science, University College London, 1997\\%            
%BEng (Hons.) Electrical and Electronic Engineering, The University of Wales, Swansea, 1992}
\school{Department of Computing Science}

%%%% In the final submission of a thesis, this should only be the year
%%%% of submission.  However, it is useful to use \date{\today} for drafts so that
%%%% they don't get mixed up.
    
\date{2022}

%% It is useful to split the document up as chapters and include
%% them.  LaTeX will sort out all the numbering and cross-referencing
%% for you --- if you run it enough times!

%% If you want to include only a couple of chapters then use the
%% \includeonly{} command with a list of the file/chapter names that
%% you wish to include.  NB, this must be in the preamble.

% \includeonly{introduction,faq}

% \def\sfthing#1#2{\def#1{\mbox{{\small\normalfont\sffamily #2}}}}

% \sfthing{\PP}{P}
% \sfthing{\FF}{F}

%% This will make sure that all cross-references are correct (including
%% references to those file not included) but will produce a dvi
%% file with only those files/chapters you specify included.

\makeglossaries

\newacronym{llm}{LLM}{Large Language Model}
\newacronym{hpc}{HPC}{High Performance Cluster}


% Commands for making in-line notes
\usepackage{xcolor}
% \newcommand{\ajs}[1]{\textcolor{orange}{[AJS: #1]}}
\newcommand{\ajs}[1]{}
% \newcommand{\note}[1]{\textcolor{red}{[#1]}}
\newcommand{\note}[1]{}
% \newcommand{\mk}[1]{\textcolor{blue}{[MK: #1]}}
\newcommand{\mk}[1]{}

\begin{document}
 
%%%% Create the title page and standard declaration.

\maketitle
\makedeclaration

%%%% Then the abstract and acknowledgements
\begin{abstract}

    We hypothesize that what we call creativity equates to an expression of a given author's unique character through their work, be it in text, speech, art, and so on. Furthermore, we claim that those can be represented as a certain set of linguistic features that can be found in text, e.g. more of this feature makes this text seem more creative.

    To that end, we explore the field of linguistic creativity through the lens of statistical text analysis. We investigate a variety of statistical measures that have been suspected to aid in text classification tasks, genre identification and others. Furthermore, we explore prior work in psycholinguistics aiming to measure the effect of words on the reader, and how those effects can be used by authors to the effect of solidifying an author's style. We also hypothesize a set of novel measures that can be linked with the understanding of structure in text. 
    
    In this work, we implement a system for working with text data, as well as methods for applying the measures we have explored. Then, we carry out an experimental evaluation of the system on a set of texts using the implemented metrics, on a broad variety of applicable NLP tasks, such as genre classification, authorship identification, and machine-generated text detection. 
    
    We conclude that the measures we have explored can be used to evaluate the creativity of text -- creativity in terms of a distinguishing feature for an author -- and how this lays out some groundwork for further research into the deep topic of computational creativity.
\end{abstract}

\begin{acknowledgements}
    Many thanks to my academic supervisor, Dr. Arabella Sinclair, for constantly being involved with the project, guiding me during my first steps in Natural Language Processing, bearing through my ramblings and entertaining my ideas.

    Many thanks to everyone who asked me questions about the project, which served to clarify and potentially solidify my ideas, as well as for giving me the opportunity to better explain my ideas to myself through explaining them to others. I will remember that one time I tried to explain this idea in terms of shepherding, sheep and shepherd dogs. 
    
    Many thanks go to the many educators who have taught me throughout the years, teaching me to think critically, rationally, and to put everything under scrutiny.
\end{acknowledgements}


%%%% It should have a table of contents, but delete the other two as
%%%% necessary.

\tableofcontents

\clearpage
\printnoidxglossary[type=\acronymtype,sort=standard, nonumberlist, title=Abbreviations]
\clearpage

%\listoftables
%\listoffigures

\chapter{Introduction}
\label{chap:intro}
\section{Motivation}
hi
% \chapter{Frequently asked questions\label{chap:faq}}

In addition to the information provided in
chapter~\ref{chap:introduction}, here are some brief notes on
references (see section~\ref{sec:references}) and figures (see
section~\ref{sec:figures}).

\section{References\label{sec:references}}

You can, of course, use any referencing style you like such as
\verb+plain+.  The \verb+natbib+ package, however, allows you to do
this with named style citations:

\begin{tabbing}
\hspace{3in} \= \kill
\verb-\citet{key}- \> Jones et al. (1990) \\
\verb-\citet*{key}- \> Jones, Baker, and Smith (1990) \\
\verb-\citep{key}- \> (Jones et al., 1990) \\
\verb-\citep*{key}- \> (Jones, Baker, and Smith, 1990) \\
\verb-\citep[chap. 2]{key}- \> (Jones et al., 1990, chap. 2) \\
\verb-\citep[e.g.][]{key}- \> (e.g. Jones et al., 1990) \\
\verb-\citep[e.g.][p. 32]{key}- \> (e.g. Jones et al., p. 32) \\
\verb-\citeauthor{key}- \> Jones et al. \\
\verb-\citeauthor*{key}- \> Jones, Baker, and Smith \\
\verb-\citeyear{key}- \> 1990 \\
\end{tabbing}

This template uses BibTeX as the citation engine, and you should learn to fill in citations correctly, perhaps even use a bibfile manager like \href{https://www.jabref.org}{JabRef}. 
Recall that when citing a book, you should always include the chapter, for example, the first chapter of the classic AI book~\cite[Ch. 1]{RussellNorvig2022}. 
Papers are easier to cite, such as a recent AAAI paper~\cite{Amado2022}. 

\section{Figures\label{sec:figures}}

To include an encapsulated postscript or PDF file (depending on whether you're using \LaTeX or PDF\LaTeX) as a figure, do
something like the following.  Note, to ensure correct
cross-referencing, it is best to include the figure label within
the caption definition.  \emph{Note that the graphicx package 
is already loaded and used to include the
University crest on the title page.}

\begin{verbatim}
\begin{figure}
 \begin{center}
   \includegraphics{myfigure.pdf}    
   \caption{This is my figure.\label{fig:mylabel}}
 \end{center}
\end{figure}
\end{verbatim}

\section{Frequently used symbols\label{sec:fus}}

In \LaTeX\ documents where you want to use a modality or some text consistently in normal text and in equation environments it is often difficult to remember to typeset the text consistently or time-consuming to keep typing in the environment. It may be a good idea to define something like the following in the preamble (i.e.\ before \verb+\begin{document}+):

\begin{verbatim}
\def\sfthing#1#2{\def#1{\mbox{{\small\normalfont\sffamily #2}}}}

\sfthing{\PP}{P}
\sfthing{\FF}{F}
\end{verbatim}

Then use it in text or math mode. In all cases it looks the same; e.g.\\
\verb+\PP\ refers to something, and other things are \FF; $\Phi = \PP\cup\FF$+\\
is typeset as:

\PP\ refers to something, and other things are \FF; i.e.\ $\Phi = \PP\cup\FF$

Note that you need to put ``\textbackslash'' after the command if you want a normal space after it.

\chapter{Related Work}
% Include background, what previous authors have done in a mostly neutral style. Optimally expecting ~1000-2000 words and 30-40 references mentioned.

\section{Challenge Landscape}

\subsection{Part of Speech Tagging}

\subsection{Word Sense Disambiguation}

\subsection{Sentiment Analysis}

% Maybe not discuss this in general
\subsection{Phonetic Analysis}

\subsection{Natural Language Generation}
\begin{itemize}
    \item top KP sampling
    \item challenges 
\end{itemize}


\section{Creative Measures}
\begin{itemize}
    \item Burstiness of verbs and derived nouns: Patterns of language are sometimes `bursty' \cite{pierrehumbert_burstiness_2012}. This paper presents an analysis of text patterns for domain X. measures include XYZ...
    \item 
\end{itemize}

\section{Tools}

\chapter{Methodology}
\label{chap:methods}

In this chapter, we explore the different datasets to be used, the methods for evaluating creativity, and the algorithms for creativity evaluation. We will furthermore discuss the strengths and limitations of the proposed methods and algorithms, including time complexity, memory constraints, and the accuracy of the results. We take an informed approach to the selection of the datasets and the methods for evaluating creativity, and discuss the reasons for our choices in detail, and support them with relevant literature as explored by other researchers in the field.

\section{Datasets}
\label{sec:datasets}
Datasets are vital for the success of any given project in the field of machine learning, and even more so when concerning linguistics. As evidenced by \cite{torralba_unbiased_2011}, the quality of the data used for training a model has a direct impact on the quality of the results. A model trained on a specific dataset, e.g. a corpus of law documents, can be expected to perform poorly on a dataset of medical documents, as the two domains are inherently different. Thus, we take particular care in planning and selecting the datasets we use. We also consider ease of use and access, as some datasets may require additional processing, others are subject to availability issues (e.g. paid datasets and corpora), and some may be too large to be used in a reasonable amount of time.
In this section, we will explore the datasets used in this project, and discuss their strengths and limitations.
\subsection{Brown Corpus}
The Brown Corpus \citep*{francis1979brown} is a widely used corpus in the field of computational linguistics, noted for the small variety of genres of literature it contains. The Corpus itself is founded on a compilation of American English literature from the year 1961. It is also small in terms of size, totalling around one million words, at least compared to modern corpora, which we also explore later on. The corpus also suffers from the issue of recency, as the works and language may be outdated for modern speakers of English.

Of interest is the fact that the corpus has been manually tagged for parts of speech, a process that tends to be error-prone. %citation good
As we will see later on, this fact has implications in terms of the supervised learning algorithms we implement for creativity evaluation. Still, we opt to utilize it primarily for prototyping purposes and drawing preliminary conclusions about the effectiveness of the implemented algorithms, rather than in-depth analysis and publication of results.

\subsection{Project Gutenberg}
Project Gutenberg\footnote[1]{\url{https://www.gutenberg.org/}} is a large collection of more than 50,000 works available in the public domain. The collection contains literature from various years and various genres and thus is suitable for training and evaluation of the developed benchmarks in the context of creativity study. 

As the Project does not offer an easy to process copy of its collection, we turn to the work of \cite{gutenberg_dataset}. The team developed a catalogue for on-demand download of the entire set of books available on the Project Gutenberg website, intended for use in the study of computational linguistics. The tool avoids the overhead of writing a web-scraper or a manual parser for the downloadable collections of Project Gutenberg books made available by third parties, as well as enables easy synchronization of newly released literature. Instead, we are only required to develop a simple pipeline for the data to be fed into the utilized systems. 

\subsection{Hierarchical Neural Story Generation}
In their work, \cite{fan_hierarchical_2018} trained a language model for text generation tasks on a dataset comprised of short stories submitted by multiple users given a particular premise (a prompt or a theme) by another user. \mk{Give an example for how one such short story would look like.} The dataset in question is technically referred to a series of posts and comments (threads) to them on the popular social media platform \textsc{Reddit}, and more tightly, the \textit{subreddit} forum \textsc{r/WritingPrompts}. The authors of the work \cite{fan_hierarchical_2018} have made the dataset available for public use, and we have used it for the purpose of evaluating the performance of our creativity benchmarks. As described by the authors on their GitHub page\footnote{\url{https://github.com/facebookresearch/fairseq/blob/main/examples/stories/README.md}}, the paper models the first 1000 tokens (words) of each story.

\mk{How do we use this dataset? You should describe the process of how we use it. }

\subsection{Discarded Datasets and Corpora}
Some datasets were considered, however, discarded due to: not being deemed applicable for the context of the application; general lack of availability of the dataset in a form that is easily accessible for our purposes; simply being infeasible to use due to the size of the dataset and the hardware constraints imposed on the project; or other reasons of similar nature.

\subsubsection*{The COCA}
The Corpus of Contemporary American English (COCA)\footnote[2]{\url{https://www.english-corpora.org/coca/}} is a large corpus of American English, containing nearly 1 billion words of text from contemporary sources. It is a collection of texts from a variety of genres, including fiction, non-fiction, and academic writing. The corpus offers a variety of tools for analysis of the data, including a concordance tool, a word frequency list, and a collocation finder. Naturally, many of those tools could be used in the field of statistical creativity analysis that we explore.

The corpus does offer limited access to the full API, as well as free samples of the data, however, the full corpus is not available for free, and the cost of acquiring it is prohibitive for the limitations set forward by the project. Nevertheless, the corpus is a valuable resource for the field of computational linguistics, and we would like to explore it further given less constraints.

\section{Word Sense Similarity}

\subsection{WordNet}\label{sec:wordnet}
WordNet\citep{wordnet1998fellbaum} is a lexical database of semantic relations between words that links words into semantic relations including synonyms, hyponyms, and meronyms. The synonyms are grouped into synsets (sets of synonyms) with short definitions and usage examples. It can thus be seen as a combination and extension of a dictionary and thesaurus \citep{enwiki:1143619785}. 

For our specific use cases, we have identified it as a valuable resource in terms of relational representation of words in semantic space. In the given context, this enables us to traverse a semantic graph for synonyms and related words for the goal of enriching potential similarity between the set of creative parts of speech (i.e., nouns, adjectives, adverbs), which we narrow down our scope to in particular. 

We can use it to look for word similarity between words. If we imagine synsets as a connected graph, usually words are similar when the distance between their synsets in terms of traversing the graph is shorter, and longer when there is no short path to reaching the other word. This is a useful property for our use case, as we can use it to determine the similarity between words in a given text, and thus, that can aid us in evaluating the creativity of the text.

% this sounds a bit weird

\subsection{Word2Vec}\label{sec:word2vec}

\citet*{mikolov_word2vec_2013} show in their work that words may be represented as dense vectors in $N$-dimensional space, and we can perform mathematical operations on them that may yield effective results in terms of word representation. What that means in our context is that we can measure similarity -- or distance -- between individual words, without knowing beforehand their part of speech of context. We can use such learned representations of words in terms of vector embeddings to determine the similarity between words in a given text.

This serves a fascinating purpose in our exploration of metrics such as surprisal that we detail further below. 

\subsection*{Measuring distance in vector representations of semantic tokens}
Intuition tells us that the dot product of vectors in $N$-dimensional space will grow when the set of vectors has similar values and decrease when the values are not similar. Thus, we can then construct the following metric for semantic similarity between vector representations of words:
$$ D(v,w) = v \times w = \sum_{i=1}^{N} v_i w_i = v_1 w_1 + v_2 w_2 + \dots + v_N w_N $$ 

The current metric, however, suffers from the problem that vectors of higher dimensions will inevitably be larger than vectors with lower dimensions. Furthermore, embedding vectors for words that occur frequently in text, tend to have high values in more dimensions, that is, they correlate with more words. The proposed solution is to normalize using the \textbf{vector length} as defined:
$$ | v| = \sqrt{\sum_{i=1}^{N}v_i^2}$$

Therefore, we obtain the following:

$$ \text{Similarity} (v, w) = \frac{v \times w}{|v| |w|} = \frac{\sum_{i=1}^{N} v_i w_i}{\sqrt{\sum_{i=1}^{N}v_i^2} \sqrt{\sum_{i=1}^{N}w_i^2}}$$

This product turns out to be the same as the cosine of the angle between two vectors:

$$ \frac{a \times b}{|a| |b|} = \cos(\theta) $$

Therefore, we will call this metric the \textbf{cosine similarity} of two words. As mentioned, the similarity grows for vectors with similar features along the same dimensions. Note the boundaries of said cosine metric: we get $-1$ for vectors which are polar opposites, $0$ for orthogonal vectors, and $1$ for equivalent vectors. Of note is the fact that such learned vector embeddings only have values in the positive ranges, thus, it is impossible to have negative values for the cosine similarity (Similarity$(a,b) \in [0,1]$).

Contrary to it, we also identify the metric of \textbf{cosine distance} between two vectors, as one minus the similarity of the vectors, or:

$$ \text{Distance}(v,w) = 1 - \text{Similarity}(v,w) $$

The cosine distance may prove useful when dealing with minimisation problems as is often the case with machine learning.

\section{Metrics}
\label{sec:metrics}

\subsection{Number of Words}
The total number of words in a given piece of text. At first glance, this metric does not impress and is, in fact, exceedingly simple. But that is fine -- we do not always need complex metrics. Sometimes, even a trivial metric as this one can inform a lot about the structure of the text. For example, the number of words in a text is directly correlated with the length of the text. This can be useful in determining the complexity of the text, as well as the time it takes to read it. In some uses, for example, comparing between books and \textit{Twitter} posts, we do not need much more information to recognize that these texts belong to entirely different genres. Such a metric is a good complement to and often used in conjunction with other metrics.

\subsection{Number of Sentences}
The number of sentences, similarly to number of words, is a trivial measure for the length of the text. However, it can be used to determine the complexity of the text. For example, a text with a large number of sentences is likely to be more complex than a text with a small number of sentences. This is because a text with a large number of sentences is likely to involve longer intellectual activity. Of course, in light of recent developments in the field of natural language generation, this metric is not particularly useful. However, due to how trivial to implement it is, it can be used in conjunction with other metrics for text classification tasks.

\subsection{Word Length}
\begin{quote}
\textit{“Because even the smallest of words can be the ones to hurt you, or save you.”} -- Natsuki Takaya 
\end{quote}
Word length fills in the set of trivial metrics we implement for text benchmarking. The intuition is simple. Given a sufficiently large corpus, the average word length -- that is, the number of characters in a word -- will converge to a certain number -- in English, this number tends to be between 4 and 5. Any deviations, either positive or negative, from this norm can then be used to determine the complexity of the text. For example, a text with a large number of long words is likely to be more complex than a text with a large number of short words. Naturally, words expressing more specific concepts tend to have a longer character length than words we use in general speech and are sometimes ambiguous. This phenomenon is established in English, although the essence may not generalize well for other languages, e.g. Chinese and Japanese, where a single character can generalize to a whole word or a concept as a whole, but given that we are working in the context of the English language, we are not concerned with this issue.

\subsection{Sentence Length}
Similar to word length above, the average sentence length is a trivial metric describing the number of characters per sentence. Intuition tells us it will be closely related to the average word length, but also indicative of text features such as complexity and readability. For example, legal documents tend to have longer sentences than, say, newspaper articles. This is because legal documents tend to be more complex and require more time to read and understand. In contrast, newspaper articles tend to be more accessible and are written in a way that is easy to understand. 

Writers may also be interested in this metric, as very long sentences are often difficult to read and understand, as the reader may lose track of the subject of the sentence among the many objects, actions and modifiers; not to mention unnecessary punctuation where simply beginning a new sentence would be far more readable... a useful feature like this can pinpoint such writing issues, inform writers where they may cut or simplify their sentences, and in general help them improve their writing style -- a feature that is often overlooked in the context of text understanding -- this is also the longest sentence in the entire document.

\subsection{Number of Tokens}
Completing the set of trivial metrics is the general number of tokens in the text. The metric correlates highly with average sentence length and word length. Rather than counting characters in the sentence or word length, however, we take a look at the number of tokens encountered in the text, usually at the sentence level. 

\subsection{Concreteness}
\label{concreteness}
Concreteness is the degree to which a word refers to a tangible object or a concrete idea. For example, the word \textit{apple} is concrete, while the word \textit{time} is abstract. \cite{brysbaert2014concreteness} provide a dataset of concreteness ratings for 40,000 English lemmas (English words and 2,896 two-word expressions (such as ``zebra crossing'' and ``zoom in''), obtained from over four thousand participants by means of a norming study using internet crowdsourcing for data collection). The dataset is based on the concreteness ratings of the four thousand participants, who rated the concreteness of 40,000 words on a scale from 1 to 5.
The concreteness of a word is measured on a scale from 1 to 5, where 1 is the most abstract and 5 is the most concrete: 

\mk{direct citation of the study, if i need to paraphrase it, probably would delete it}
\begin{quote}
    Some words refer to things or actions in reality, which you can experience directly through one of the five senses. We call these words concrete words. Other words refer to meanings that cannot be experienced directly but which we know because the meanings can be defined by other words. These are abstract words. Still other words fall in-between the two extremes, because we can experience them to some extent and in addition we rely on language to understand them. We want you to indicate how concrete the meaning of each word is for you by using a 5-point rating scale going from abstract to concrete.
\end{quote}

The dataset provides norms for the 40,000 words and 2,896 two-word expressions -- including mean and standard deviation for each entry.

The intuition of this metric is that a word that is more concrete is more likely to be used in a creative context, as it is easier to imagine and relate to. It not only describes one aspect of the word's meaning, but authors (and genres, in general), tend to exhibit specific characteristics, such as legal documents being more generally more concrete - one would expect concrete objects and entities to appear more in documents such as the UN Human Rights Charter, or protocols for health standards control, for example. 

\subsection{Imageability}
\label{sec:imageability}
Imageability is the degree to which a word evokes a mental image, as described by \cite{degroot1989representational}. For example, the word \textit{apple} is more imageable than the word \textit{time}. \cite{cortese_imageability_2004} provide a dataset of imageability ratings for 3,000 single-syllable English lemmas, obtained from over four thousand participants by means of a norming study using internet crowdsourcing for data collection. The dataset is based on the imageability ratings of the four thousand participants, who rated the imageability of the 3,000 words on a scale from 1 to 7. The dataset also contains the number of participants who rated each word, the standard deviation of the ratings, the mean and the standard deviation for the reaction time of the participants.

\subsection{Frequent Word Usage}
\label{frequency}

\begin{quote}
\textit{    “Separate text from context and all that remains is a con.”} -- Stewart Stafford 
\end{quote}

Word frequency refers to the number of times a given word appears in a given context. Word frequency naturally differs from text to text, and smart word choice in general is an excellent indicator for intellectual linguistic use. The intuition behind selecting this metric is that words that are occurring less frequently in common speech are more likely to be used in a creative context. To give an example by rewording the last sentence, would yield: ``The intuition behind identifying this linguistic measure owes to the words' property of inverse proportionality between frequency and perceived creative or intellectual value.''

As noted, less common words are associated with higher perceived intellectual value. Even more so, the use of less common collocations (words occurring very close in a given context) hints at a higher level of linguistic skill. Of course, simply chaining completely unrelated words together (e.g. ``palmarian tobaccophile ephemeron urbarial'') hints not to high intellectual value, but rather to spitting out a random sequence of words. Properly applied in context, though, commonly not associated words can be used to great effect. This is especially true in the case of poetry, where the use of uncommon words and collocations is a common practice, or, for example, in biological contexts, such as medicine and botany, where very precise yet niche namings and conventions are mandated. This type of dissonance between common speech and niche terminology is a common theme in creative writing, and is often used to great effect. For example:

\begin{quote}
    \textit{``When they'd gone the old man turned around to watch the sun's slow descent. The Boat of Millions of Years, he thought; the boat of the dying sungod Ra, tacking down the western sky to the source of the dark river that runs through the underworld from west to east, through the twelve hours of the night, at the far eastern end of which the boat will tomorrow reappear, bearing a once again youthful, newly reignited sun.''}
    \begin{flushright}
        -- \textit{The Anubis Gates}, Tim Powers
    \end{flushright}
\end{quote}
In this context, ``boat'' is a completely valid and understandable synonym of the word ``sun'', yet the word ``boat'' co-occurring with the word ``sun'' outside this context is not common, and therefore, we are prompted to believe that this context is more `creative'.

We tackle the topic of contextual surprise further on with subsequent metrics, but for now, we focus on the general idea of individual word frequency. 

Given a sufficiently large linguistic corpus, we obtain a list of words and their frequency of occurrence. We can then use this list to calculate the frequency of occurrence of a given word in a given text. We can then use this frequency as a metric for the text's creativity. Choice of corpus is key here, as the corpus should be large enough to contain a wide variety of words, but not specialized enough to inflate the frequency of niche words. For example, a corpus of medical texts would contain a lot of medical terminology, which would inflate the frequency of medical terms, and therefore, would not be a good choice for a general creativity metric, for example in the case of a poetry contest.

For our use case, we opt to use the British National Corpus (BNC) \citep{bnc-20.500.14106/2554}, which is a 100 million word collection of samples of written and spoken language from a wide range of sources, designed to represent a wide cross-section of British English from the later part of the 20th century, both spoken and written. The BNC is a good choice for our use case, as it is a general corpus, and contains a wide variety of words, but is not specialized enough to inflate the frequency of niche words. 

The frequency lists we use are provided by the work of \cite{leech_rayson_wilson_2014} and are readily available in sheet form for both lemmatized and non-lemmatized words. In our case, we attempt to adhere only to the lemmatized versions in order to have consistency with previous metrics, but also to have normalized results, e.g., although the words `am', `is', `are' are all inflections of the verb `to be', they may have different frequencies and different positions in the list. POS tagging and lemmatization again come into play here, as we need to be able to identify the lemma of a given word in order to find its proper frequency in the list. The frequency lists indicate the words' frequencies per 100 million tokens. Intuitively, given a varied enough corpus such as the BNC, we expect these numbers to normalize and generalize well for general English. We then use the frequencies for the lemmas and the take the logarithm with base 10 of the given frequency like so:

\begin{equation}
    \label{eq:frequency}
    \text{freq}(x) = \log_{10}(\text{Frequency}_{BNC\ 1M}(\text{Lemma}(x)))
\end{equation}

Like before, if a word does not appear in the BNC, we discard it and continue. We then calculate the average frequency of the words in the text, and return the metric for interpretation by the end user.

\subsection{Proportion of Parts of Speech}
\label{pos_prop}

\begin{quote}
\textit{    “The difference between the right word and the almost right word is the difference between lightning and a lightning bug.”} -- Mark Twain
\end{quote}

We hypothesize that the proportion of part of speech tags can have meaning in terms of distinguishing genre and author characteristics. For example, a text with a larger proportion of nouns is likely to be inherently different to a text containing a larger proportion of verbs or adjectives. In fiction and creative writing, we are likely to see a higher prevalence of nouns and verbs, as those tend to describe abstract concepts, or to portray some scene with characters. Legal documents may be more heavily reliant on having more nouns and adjectives, as they are more likely to be used in a descriptive context. Internet forums and social media posts are likely to have a higher proportion of verbs, as they are more likely to be used in a conversational context, and attention is more likely to be focused on the action rather than the object. In general, we expect the proportion of parts of speech to be indicative of the genre of the text, and therefore, we can use it as a metric for creativity and classifying genre.

We use the NLTK library \citep{nltk_citation} to perform part of speech tagging on the text. We then calculate the proportion of each part of speech tag in the text, and return the metric for interpretation by the end user. For most purposes, we use the universal tagset, which makes no distinction, for example, between common nouns such as `person' or `city', and proper nouns such as `Alice' and `London', and instead groups them together under the tag `NOUN'. We encourage more ambitious authors to explore the possibilities of using the more fine-grained tagsets, such as the Penn Treebank tagset.

\subsection{Predictability}\label{predictability}\label{heavyweight_metrics}

\begin{quote}
\textit{    “The most exciting phrase to hear in science, the one that heralds new discoveries, is not `Eureka!' but `That's funny...'}” -- Isaac Asimov
\end{quote}


One aspect of creativity relates to how we perceive novelty in using already existing concepts. Imagine a person who has only ever read and knows a single book. For them, any other book they read after that has the potential to be infinitely more different from the first book they know. However, a person who has read a thousand books, is likely to not perceive much novelty in most of the text of the thousand and first book they get to read. Yet, this aspect of novelty is precisely what some people may consider to be creative -- proposing novel solutions to an existing or even a non-existing problem.

Consider again the example of the person who has read a single book and the person who has read a thousand books. In a sense, this aspect of value and novelty is completely subjective and dependent on the individual's prior knowledge. However, common sense is something that a majority of the people have. This common sense means that there must be some unifying factor between people that makes them perceive novelty in a similar way. We hypothesize one such factor to be what we call the predictability of the text.

Predictability refers to how predictable words are in a given context. For example, given the sentence ``Jenny was feeling sick, so Jenny went to the \dots'', one would commonly expect the next word to be ``doctor'', as it is the most predictable word in this context. However, if the next word is ``beach'', we could say that the sentence is more creative, as it is less predictable. Naive intuition tells us that the more predictable a word is, the less creative it is. Of course, naive is the key word here, as we make some overly simplifying assumptions. Exceptions to this rule will apply, but we believe that this generalizes well for the majority of cases. 

Then, the common question would be how to detect this kind of predictability. In the trivial case, we can hypothesize a bigram, trigram, or some n-gram model that would be able to predict the next word in a given context. This n-gram model will have learned some probability distribution of the tokens, that is, given the context:
\begin{equation}
    P(\text{word}| \text{context}_1, \text{context}_2, \dots, \text{context}_{n-1}) = \text{probability of the word given the context}
\end{equation}

It will return some probability distribution for the concrete word given the context length $1, 2, \dots, n-1$. 
We can then use this probability distribution to calculate the predictability of the text. For example, given a text, we can calculate the probability of each word in the text given the context of the previous $n-1$ words. We can take the average of these probabilities, and return the metric for interpretation by the end user. That would be a similar approach to how we calculated the concreteness and imageability ratings. But we can do better, given the state-of-the-art.

Recent Large Language Models (LLMs) have shown ability to understand text even within a vast context and Transformer-based models in particular have been performing excellently in tasks such as text summarization, text generation, and question answering. We specifically turn our attention to a particular class of LLMs, called Masked Language Models (MLMs). Those are trained to predict a masked token in a given context. Given the sentence above, the model would be trained by masking each of the words in the sentence, and then predicting the masked word given the context. This leads to a model capable of understanding context to an impressive degree and predicting the next word in a given context.

We put forward the intuition that one such LLM, if trained on a broad enough subset of the English language, would be able to represent something akin to ``common sense'', in terms of the linguistic capabilities of the so-called `averaged' human.
Then, if we put the said model to work by trying to predict a word given some context, it could represent the intuition of the average human in terms of what the average human would expect to see. Therefore, instead of limiting ourselves to just some memory-hungry n-gram model, we can use a pre-trained LLM to calculate the predictability of a word within a context.

Then, how would we formulate predictability of a given word, given some context, in terms of LLMs? \textbf{We define predictability as the metric of averaged gradient over the likelihoods of the top K masked token replacement suggestions}. 

When we mask a word in a given context, the model can return a probability distribution of the most likely words to replace the masked token with, along with likelihoods for those words. We can take the list of the top K likelihoods and sort it into descending order for the values. We can then take the gradient of this list to obtain values for how the likelihood changes over the order of the likeliest words. A probability distribution that is more uniform will have a smaller gradient, while a probability distribution that is more skewed will have a larger gradient. We can then take the average of the gradient values, and return the metric for interpretation by the end user. 

In a way, predictability is a measure of the model's confidence in its predictions. A text, for which the model is more confident in its predictions, is likely to lead to less creative text, as it is more likely to predict the most likely word in a given context. On the other hand, a text, for which the model is less confident in its predictions is likely to lead to more creative text, as it is more likely to predict a less likely word in a given context.

\subsection{Surprisal}\label{surprisal}

\begin{quote}  
\textit{"Nothing is more dangerous than an idea when it is the only one you have."} -- Emile Chartier
\end{quote}

Surprisal is a metric that is closely related to predictability. In fact, surprisal follows closely the outlined algorithm for obtaining masked word suggestions given some context. The difference is that, whereas predictability deals with the model's confidence in suggesting (or, more specifically, the probability distributions), surprisal deals with the actual tokens that have been suggested. 

\textbf{We define surprisal as the averaged similarity of a given word to the top K masked token replacement suggestions.} 

Following the procedure above, we can obtain the top K masked token replacement suggestions for a given word in a given context. We can then calculate the similarity of the given word to each of the top K masked token replacement suggestions. We can then take the average of the similarity values, and return the metric for interpretation by the end user. We tackled the topic of word sense disambiguation back in Chapter \ref{chap:background} and Sections \ref{sec:wordnet} and \ref{sec:word2vec}, so we will not delve into the details of what we define as similarity here, but we use approaches in the resources above.

The name surprisal may not be very apt for the metric, as it is not a measure of how surprised a human would be to see a given word in a given context, but nonetheless it relies on the intuition that a word that tends to have nothing in common with the top K masked token replacement suggestions is likely to be more surprising than a word that tends to have a lot in common. This kind of surprise can provide hints as to potential aspect of creativity in text.

\chapter{Design}
\label{chap:design}

In the following chapter, we introduce concepts behind the design of the developed library and how those will be implemented in the application. We also discuss the technology choices we have made and the reasoning behind them. Furthermore, we take a look at how the application will be structured and how it will be distributed, and how the users will be able to interact with it via a command-line interface, or apply declared methods and classes inside their own applications. Finally, we discuss the delivery of a documentation and a user guide, as well as the testing and validation of the application.

% We might want to consider the task in the context of software development as well
\section{Requirements}
In this section, we detail the requirements regarding the application and the library in the following sections. We will also discuss the requirements regarding the documentation and the user guide, as well as the requirements regarding the testing and validation of the application.

\subsection{Scenarios}
We use scenarios to better illustrate the use-cases of the \textsc{Mad Hatter} package. They provide a frame for the functional and non-functional requirements and allow us to nail down the specific requirements and pain points, well in advance of the de facto launch of the package.

\begin{enumerate}
    \item Evaluating linguistic features associated with creative aspects of language in a given text. Users should be able to clearly present a text and evaluate it against a set of metrics, which will then be presented in a clear and concise manner.
    \item Providing methods for interacting, plotting and visualizing the results of the evaluation. Users should be able to easily interact with the results of the evaluation, and plot them in a way that is easy to understand and interpret.
    \item Providing a pipeline for batch data analysis to be used in larger-scale linguistic research operations. The users should be able to easily process large amounts of data in a batch manner, and then interact with the results of the evaluation in a way that is easy to understand and interpret.
\end{enumerate}

\subsection{Functional Requirements}
Functional requirements pinpoint the specific tasks that the application needs to be able to perform. They are the most important part of the requirements, as they are the ones that will be used to evaluate the success of the application. We list those below.

\textbf{(FR1) The user must be able to evaluate a given text against a set of metrics.}

When fed a text, the system must return evaluation of the text against a set of metrics. The user must be able to specify which metrics they want to use for the evaluation, and the system must be able to return the results of the evaluation in a clear and concise manner.

\textbf{(FR2) The user must be able to interact with the results of the evaluation.}

The user must be able to interact with the results of the evaluation in a way that is easy to understand and interpret. The user must be able to plot the results of the evaluation, as well as export them in a format that is easy to use in other applications.

\textbf{(FR3) The user must be able to batch-process a large volume of texts.}

The user must be able to batch-process a large volume of texts. The return format of the data must be easy to integrate with other applications, especially in the context of linguistic data analysis.


\subsection{Non-functional Requirements}
The package henceforth needs to satisfy a list of viable non-functional requirements, which we will list below.

\textbf{(NFR1) The package must be able to be used by users with minimal technical knowledge.}

The package should provide methods that make text analysis viable for users with minimal technical knowledge. Beyond installation, it should be trivial to evaluate and interpret the results of the evaluation. \mk{addressed by python!} The package must also be able to be used with all sorts of texts --- from single sentences to full-length books.

\textbf{(NFR2) The package must work within limited computing capabilities.}

As above, users should be able to run and evaluate their texts on their personal computers, without the need for specialized hardware. The package should be able to run on a single machine, and should not require any specialized hardware.

\textbf{(NFR3) The package must be able to scale to large amounts of data.}

The package should be able to scale to large amounts of data, and should be able to process large amounts of data in a reasonable timeframe. 

\textbf{(NFR4) The package must be easily configurable.}

The package should be easily configurable, and should allow users to easily change the parameters of the evaluation. The package should also allow users to selectively include metrics to the evaluation.

\textbf{(NFR5) The package must be easily extensible.}

The package should be easily extensible, and should allow advanced users to easily add new metrics to the evaluation. The package should also allow users to easily add new methods for interacting with the results of the evaluation.

\textbf{(NFR6) The package must be well-documented.}

The package should be well-documented, and should provide a clear and concise user guide, as well as a clear and concise documentation for the package itself. The documentation should be easily accessible and should be easy to understand.

\textbf{(NFR7) The package must be issue-proof.}

The package should be well-tested against a set of test cases, and must contain as few bugs and unhandled behaviours as possible. Exceptions and errors, wherever they may occur, must have a clear and concise message, guiding users to the source of the issue.

\subsection{Concerns}

Working with text and developing a package for text analysis results in several concerns we need to address. 

\paragraph{Speed.} Working with text can be \textit{slow}. This is especially true when working with large amounts of text, or when working with large language models. This means that we must take into account the algorithms and data structures we apply, as a single unoptimized algorithm can result in a very slow application. We must also take into account the size of the language models we use, as larger models tend to be slower to process.

\paragraph{Memory Consumption.} The old adage that \textit{memory is cheap} is not entirely true. While it is true that memory is cheap, it is also true that memory is not free (\textit{and no, we cannot ``just download more RAM''}). Furthermore, model accuracy tends to grow with the size of the neural network and the size of the used vocabulary. Naturally, we then need to seek a compromise on the size of the models we use, as we cannot:
\begin{enumerate}
    \item Feasibly make use of the larger model variants during the research stage of the project, where we aim to process large corpora, evaluate the performance of the algorithms on them and make conclusions about the data. If we do aim to speed up this process, we can benefit from parallel computing --- but processing large batches of text in parallel has a non-negligible likelihood of running out of allocated memory even on some \acrfull{hpc} clusters.
    \item Expect users to run too large models on their personal computers, as this would result in a very poor user experience and a far reduced space of potential users. We do not plan to hardcode any models (large or small) in the application, however, the provided guides will reference selected smaller-scale pretrained LLMs, which come with the advantage of being more sustainable long-term. Naturally, we will also provide a way for more experienced and more capable organizations or individuals to run larger models with minimal effort. 
\end{enumerate} 

\paragraph{Ease of Use.} The target users are not expected to be very technically proficient. This means that we need to provide a way for users to easily interact with the application, and easily interpret the results of the evaluation. We also need to provide a way for users to easily install and use the application, without the need for specialized hardware or software. The application should be potentially viable for instant feedback in the context of writing assistance, and potentially integrable with existing writing tools.

\section{Technology Choices}
\subsection{Python}
We will be using the Python\footnote{\url{https://www.python.org/}} programming language for the development, prototyping, and release of the system. Python is a mature dynamically-typed interpreted programming language with a rich ecosystem of libraries and frameworks, especially popular with academic staff and data scientists. Python fundamentally lags behind the competition in terms of raw speed and performance, but makes up for it with its ease of use and rich ecosystem. Wherever performance is desired, library developers instead implement core code in a more-performant language, and instead provide bindings for Python, thus making Python a viable choice even in terms of computing-heavy tasks. 

Specifically, we use Python version 3.10.X as shipped by the Anaconda software package. We are aware that the most recent stable version of Python 3.11 brings non-negligible optimizations and faster execution speed for some Python scripts, however, in light of the fact that the Anaconda distribution is still shipping Python 3.10.X, and the fact that some packages have not been well-tested with Python 3.11, we opt to use that version for the time being. We will be using the Anaconda distribution as it is a very popular and mature distribution of Python, which is also very easy to install and use. It also comes with a large number of pre-installed packages, which will be very useful for the current developer experience.

\subsection{NumPy}

NumPy\footnote{\url{https://numpy.org/}} is a Python library for scientific computing. NumPy provides a variety of highly optimized data structures, as well as a large number of mathematical functions that can be applied to said data structures. It is a very popular library, and is widely used in the Python ecosystem. NumPy boasts a very mature library, with a large community of developers and researchers, leading to a very well-tested and well-documented product. Furthermore, because the core of NumPy is implemented in low-level programming languages, such as the C programming language, the performance of functions using methods in NumPy barely lags behind for math-heavy operations against even the fastest languages available. 

NumPy addresses our fundamental need for a performant library for scientific computing. Implementing specific array data structures in NumPy enables us to severely cut down on performance time for math-heavy operations, as well as cut down on memory consumption. NumPy also interfaces easily with PyTorch, another library for high-performance computing, which we will be using for the implementation of the models used in the application.

\subsection{PyTorch}
PyTorch is Python library for imperative-style high-performance computing, optimized for parallel operations on the GPU, and particularly applied for deep learning \citep{NEURIPS2019_9015_pytorch}. PyTorch is an open-source library with rich documentation, and many deep learning models have been implemented with it. The functions and models available within PyTorch are essential for the implementation of the heavyweight metrics we choose to implement. 

Additionally, many of the freely available \acrfull{llm} architectures have already been implemented and are publicly available via high-level APIs interacting with the models. Thus, we can avoid implementing or reimplementing the LLM architectures on our own, and instead focus on the applications of the implemented package. Furthermore, any LLMs used in the application will be implemented in PyTorch.

\subsubsection*{\textbf{Comparison with TensorFlow}}
TensorFlow is a Python library for the same purpose as PyTorch -- to provide low- and high-level abstractions for efficient parallel computing, optimized for GPU operations, and implementations of deep learning scientific models. Anything one can achieve with PyTorch, they can achieve with TensorFlow, as well. Similarly, most of the common model architectures are also available as high level abstractions implemented in TensorFlow. In general, there are very few key differences between the two.

Ultimately, however, we have opted to use PyTorch over TensorFlow for the following reasons:

\begin{enumerate}
    \item PyTorch is more popular in the academic community, and is more widely used in research. This means that there is a larger community of developers and researchers working with PyTorch, and a larger number of models implemented in PyTorch. 
    \item Integration with NumPy. Because key computations in the project will be handled by NumPy, it is important that the library we use for high-performance computing integrates well with NumPy. PyTorch integrates very well with NumPy, and is able to convert NumPy arrays to PyTorch tensors with minimal effort.
    \item PyTorch is more ``Pythonic'' than TensorFlow. This means that PyTorch is more intuitive to use, and the general coding style is more natural -- leading to an excellent developer experience.
\end{enumerate}

\subsection{NLP Frameworks}
In this section, we evaluate a variety of available Python libraries for \acrfull{nlp} tasks, and discuss the reasoning behind our choice of the library we will be using in the application.

\subsection*{NLTK}
NLTK is a key Python library for natural language processing, primarily built for education purposes and managed as an open-source software, built to be relatively modular and lightweight. Commonly used by researchers and students for understanding and implementing algorithms for NLP tasks, it is a relatively popular and mature framework with a healthy extension ecosystem, where contributors are able to write their own modules and share them with the community.  

The strengths of NLTK include: 
\begin{itemize}
    \item A large number of modules for a variety of NLP tasks, including tokenization, stemming, tagging, parsing, and so on.
    \item A large number of corpora and datasets for a variety of NLP tasks, including the Brown Corpus, the Penn Treebank, and so on.
\end{itemize}

\subsection*{TextBlob}
TextBlob is an NLP library drawing heavy inspiration from algorithms available in NLTK, providing a higher-level abstraction for common NLP tasks. It is built to be easy to use, and provides a very intuitive API for common NLP tasks. It provides an easy interface for parsing and working with text on all levels, but it is somewhat lacking in terms of more advanced NLP tasks. Furthermore, development has stagnated and is rarely updated with new features or bug fixes.

\subsection*{SpaCy}
SpaCy is an open-source Python library for advanced natural language processing, designed to be easily applied in production environments and implementing pipelines for enhanced NLP tasks. Whereas NLTK is primarily used for research and education, SpaCy is commonly being applied in industry environments. SpaCy is in active development and enjoys a large community of developers and researchers.

The strengths of SpaCy include:
\begin{itemize}
    \item pretrained pipelines currently supporting tokenization and training for 70+ languages
    \item state-of-the-art speed and neural network models for tagging, parsing, named entity recognition, text classification and more
    \item multi-task learning with pretrained transformers like BERT, as well as a production-ready training system and easy model packaging, deployment and workflow management.
\end{itemize}

\subsection*{Comparison between NLTK and SpaCy}

We draw a few differences between NLTK and SpaCy in the context of our application. We ignore TextBlob, as it has most of the capabilities of NLTK but seems to have stagnated in development. The two frameworks were compared across the following criteria:

\begin{enumerate}
    \item \textbf{Ease of Use.} Both frameworks are relatively easy to use, but SpaCy is built as an NLP Swiss army-knife not just for production, but all tasks. NLTK, on the other hand, requires some reading and understanding of the underlying algorithms to be used effectively.
    \item \textbf{Performance.} Both frameworks perform relatively similarly in performance for key algorithms such as tokenization and tagging. A lot of the SpaCy key code has been implemented in Cython, a static compiler for Python providing magnitudes of speed improvements over regular Python, in theory making it faster for certain applications. We do not account for the performance of text processing pipeline modules using neural networks, as we will not be using those in the application. In the limited testing we did, we did not find a significant advantage for the simple use case we had for an NLP framework. Therefore, this criterion is a tie. 
    \item \textbf{Memory Consumption.} SpaCy inherently requires more memory as it does much more than tokenization internally and stores more information about the text. Modules utilizing NLTK algorithms, on the other hand, are more lightweight and can be selected to use less memory. Table \ref{tab:spacy_vs_nltk} lists a small-scale experiment we carried out to compare the memory consumption of the two frameworks.
\end{enumerate}

It is unlikely that we will use all capabilities that both of SpaCy and TextBlob provide straight out from the box. We can, therefore, avoid the overhead of the other two frameworks, and instead use NLTK and choose which modules we use selectively. Thus, our NLP framework of choice is NLTK.

\begin{table}[htbp]
    \centering
        \begin{tabular}{lrr}
            \toprule
            Framework  &  Peak Memory  &     Increment \\
            \midrule
                SpaCy  &  5089.13 MiB  &   4465.29 MiB \\
                NLTK &  434.81 MiB   &     48.75 MiB \\
            \bottomrule
            \end{tabular}
    \caption{We carried out a small experiment to compare the memory consumption of SpaCy and NLTK. We prepared a subset of 1000 book files and only opted to tokenize and tag the words of each, and store both as simple Python lists. We ran the experiment in parallel with multiple ($n=16$) processes, to simulate a real use case we would have for the frameworks. A Python process running SpaCy, unfortunately, even when stripped down to a minimal pipeline, consumes massive amounts of memory. We can see that the memory consumption of SpaCy is almost 10 times that of NLTK. }
    \label{tab:spacy_vs_nltk}
\end{table}


\section{Code Style}
\subsection{PEP8}
We will be using PEP8\footnote{\url{https://peps.python.org/pep-0008}} \citep{pep8} as our code style guide. PEP8 is a style guide for Python code, which is maintained by the Python Software Foundation. It is a very popular, and comprehensive style guide, widely used by many Python developers and organizations. It covers a wide range of topics, including naming conventions, indentation, line length, whitespace, comments, ``docstrings'' (short for documentation strings, or, more specifically, comments that explain the way a given procedure or a class works, inside the code itself), and so on. Furthermore, we encourage the usage of type hints in function definitions as well as the usage of type annotations in docstrings.
\subsection{Docstrings}
We use NumPy's code style guide for documenting classes and functions\footnote{\url{https://numpydoc.readthedocs.io/en/latest/format.html}}. Using NumPy's style guide provides a clear and concise way of documenting the parameters and return values of functions and methods. It also has the added benefit of being familiar with other developers and researchers, as NumPy is a very popular library in the Python ecosystem, therefore potential users will be familiar with IDE-provided Intellisense help pop-ups on function definitions. Finally, it is handy as potential usage of a static documentation generator such as Sphinx\footnote{\url{https://www.sphinx-doc.org/en/master/}} is made easier, as Sphinx is able to parse NumPy-style docstrings and generate documentation from them.

\subsection{Linting}
As part of the development process, we set up an automated linting service inside the IDE using the Pylint linter \footnote{\url{https://www.pylint.org/}}. Pylint checks for errors in the code, as well as enforces the PEP8 code style guide. It also provides a variety of other checks, such as checking for unused imports, checking for unused variables, checking for undefined variables, and so on. We also use the Pylance language server\footnote{\url{https://marketplace.visualstudio.com/items?itemName=ms-python.vscode-pylance}} for the Visual Studio Code IDE, which provides a variety of other checks, such as type checking, and so on.

\subsection{Testing}


\subsection{Code Review}
\subsection{Version Control}
The outlined project 
\subsubsection{Git and GitHub}

\section{Documentation}
We use Sphinx in this household.
\subsection{Sphinx}

\subsection{Hosting}



\chapter{Implementation}
\label{sec:implementation}

\chapter{Evaluation}
\label{sec:evaluation}
In the context of application, we opt to implement several experiments that give a better understanding of the potential applications of \textsc{Mad Hatter}. 

\begin{table}[htbp]
    % latex table with the following specs: i9-9800h, 16gb ram, MacOS
    \centering
    \begin{tabular}{lp{0.6\textwidth}}
        \toprule
        \textbf{Component} & \textbf{Description} \\
        \midrule
        \textbf{CPU} & 2.3 GHz Intel Core i9-9800H \\
        \textbf{RAM} & 16 GB DDR4 2400 MHz \\
        \textbf{GPU} & Intel UHD Graphics 630 / Radeon Pro 560X  \\
        \textbf{OS} & macOS 13.1\\
        \bottomrule
    \end{tabular}
    \caption{Specifications of the computer used for the experiments.}
    \label{tab:specs}
    
    
\end{table}
Wherever not explicitly mentioned, assume the specifications listed in Table \ref{tab:specs}.

\section{Experimental Design}
\label{sec:experimental_design}
In this section, we describe the experiments we conducted to evaluate the performance of \textsc{Mad Hatter}. We start by describing the datasets we used for the experiments. Then, we describe the experiments we conducted and the metrics we used to evaluate the performance of \textsc{Mad Hatter}. Finally, we describe the baselines we used for comparison.

\subsection{Datasets}
\label{sec:datasets_expdesign}
Table \ref{tab:used_datasets} describes the utilized datasets along with their specific application in the experiment. Further descriptions of the datasets can be found at section \ref{sec:datasets}.

\begin{table}[htbp]
    \centering
    \begin{tabular}{ll}
        \toprule
        Experiment & Dataset(s) \\
        \midrule
        Document Class Identification & 1. Project Gutenberg (PG) \\
        & 2. EU DGT-Acquis \& Europarl Corpus [NLTK] (Legal) \\
        & 3. r/\textsc{WritingPrompts} (WP) \\
        \midrule
        Authorship Identification & Project Gutenberg (PG) \\
        & Up to 30 works from the 1000 most prolific authors \\
        \midrule
        Machine-Generated Text Detection & 1. WebText (representing real text)  \\
        & 2. Generated texts from GPT-2 XL-1542M\\

        
        \bottomrule 
    \end{tabular}
    \caption{Listing with the datasets used for the experiments.}
    \label{tab:used_datasets}
\end{table}

\mk{note sometimes that gpt-2 texts are generated from the given training data}

\section{Experiments}
We implement three different experiments as a way of evaluating the performance of the application. The first experiment is a document class identification experiment, where we evaluate the performance of \textsc{Mad Hatter} in identifying the class of a document, thus demonstrating that the features we implement are well-defined and enable differentiating between different types of writing. We then move on to evaluating how well the algorithm can differentiate between different writing styles, a task also known as authorship identification. Finally, we follow the logical progression of authorship identification to address a topic that has been gaining traction in recent years, that of machine-generated text detection. This may have further applications in the future, as the field of natural language generation has been steadily growing in the past few years, with the advent of LLM such as GPT-2 \citep{radford2019_gpt2} and GPT-3 \citep{brown_gpt3_2020}.  

\subsection{Document Class Identification}
\label{sec:document_class_identification}
In this experiment, we evaluate the performance of \textsc{Mad Hatter} in identifying the class of a document. The datasets, described in Table \ref{tab:used_datasets}, form the basis of the classes we designate, those being: (conventional) fictional literature (Project Gutenberg / PG), legal texts from the EU DGT-Acquis as well as the Europarliament Corpus distributed with NLTK (Legal / LG), and short-form stories from the subforum \textsc{WritingPrompts} of the social media platform \textsc{Reddit}(WP). 

\subsection*{Setup}
Initially, all distinct texts are split into chunks of 100,000 characters (with the trailing chunk on its own). This is done primarily to maximize the potential data points of the dataset, but also to speed up the processing of the algorithm for large texts (for example, the texts in PG dataset are usually long-form full books which have upwards of 600,000 characters, assuming a ratio of 100,000 characters per 60-70 pages of text in traditional font and size). Normally, this may carry a potential for overfitting, as the chunks may not be representative of the whole dataset. However, as the texts are 1) very distinct from each other, and 2) have been shown to not split to more than 6-7 chunks, this is not a concern. 
The datasets are run through a simple pipeline that generates the features described in Section \ref{sec:metrics}. For more flexibility in combining and comparing the datasets for classification, each dataset is separately run through the pipeline. After the features are extracted, each dataset is assigned its respective category. The datasets are then combined and shuffled.

The combined dataset is split into a training, validation, and test set, with a ratio of 80:10:10. The training set is used to train a logistic regression with L2 penalty, which is then used to predict the class of the documents in the test set. As an intermediary step, we run a grid search with the training dataset and the validation dataset in order to find the best parameter for the inverse of regularization strength of the algorithm. The parameter is chosen from the set $\{\frac{1}{64}, \frac{1}{32}, \frac{1}{16}, \frac{1}{8}, \frac{1}{4}, \frac{1}{2}, 1, 2, 4, 8 , 16, 32 , 64\}$. The parameter with the highest accuracy on the validation set is chosen for the final model. The accuracy of the model is then evaluated on the test set.

\begin{table}[htbp]
    \centering
    \caption{Performance results for Document Classification}
    \label{tab:document_classification}
    \begin{tabular}{ll}
    \toprule
    Experiment & Document Classification \\
    \midrule
    Size of Train Set & 4686 \\
    Train Accuracy & 99.827\% \\
    Validation Accuracy & 99.808\% \\
    Test Accuracy & 99.827\% \\
    \bottomrule
    \end{tabular}
    \end{table}

\subsection*{Results}
Table \ref{tab:document_classification} shows the performance results for the document classification experiment. The results show that \textsc{Mad Hatter} is able to identify the class of a document with a very high accuracy. This is not surprising, as the classes are very distinct from each other, yet it affirms that the implemented features capture well specific characteristics of the text. The results also show that the model is not overfitting, as the accuracy on the test set is very similar to the accuracy on the training set.

It should be noted that, despite the size of the training set is relatively small as opposed to other experiments in the field of document classification, the accuracy achieved is remarkably high. This is due to the fact that the features used are very simple and straightforward, and thus do not require a large amount of data to be learned. Furthermore, the algorithm is a step-up in terms of speed from existing baselines such as SVMs and TF-IDF algorithms, which makes it more suitable for large datasets and big scale text analysis. Figure \ref{fig:cmatrix_document_classification} shows the confusion matrix for the document classification experiment. As seen, the document is able to distinguish between the classes with an excellent accuracy, precision and recall.

\subsection*{Discussion}
Via the algorithm, the classes have been shown to not only be evidently distinct on their own, but also in terms of the features used. The features used in the experiment are very simple and straightforward, and thus do not require a large amount of data to be learned. Potential applications for document classification may include categorizing documents in a large database or potential dataset. Categorization can possibly be applied for sentiment evaluation for product reviews, social media posts, and so on. We go on to explore other potential uses of the algorithm in the following experiments.

\begin{figure}[htbp]
    \centering
    \includegraphics[width=0.5\textwidth]{../src/plots/document_classification/heatmap.png} 
    \caption{Confusion Matrix for Document Classification. The rows represent the true labels, while the columns represent the predicted labels.}
    \label{fig:cmatrix_document_classification}
\end{figure}

\subsection{Authorship Identification}
\label{sec:authorship_identification}
After we have identified that the features are able to distinguish between different classes of documents, we now ask, ``Can a machine distinguish between texts from the same class?'' 

This is a homogenous classification task, where the classes are very similar to each other, and the specific task is to identify the author of a text, given a set of candidate authors. 
For example, given the text of ``Alice in Wonderland'', \textsc{Mad Hatter}'s task is to identify that the author is Lewis Carroll. The task is a natural progression from the document classification task and a more fine-grained existing problem in the field of NLP. 

\subsection*{Setup}
We make use of the work done by \cite{gutenberg_dataset} to standardize the Project Gutenberg for data exploration and analysis. We filter out for the most prolific 1000 authors in the available non-copyright literature. Furthermore, we randomly sample a maximum of 30 works per author. This is done in order to avoid overfitting for the more prolific authors (number one has more than 300), and even then only the top 200 authors have more than 30. A single chunk of 100,000 characters is then taken from each text and added to a list for processing. 

The pipeline is similar to the one listed for document classification. We process all samples and obtain the features described in Section \ref{sec:metrics}. The dataset is then split into a training, validation, and test set, with a ratio of 80:10:10. The features are then standardized (subtracting the mean and dividing by the standard deviation for each column). The training set is used to train a logistic regression with L2 penalty, which is then used to predict the author of the documents in the test set. As an intermediary step, we run a grid search with the training dataset and the validation dataset in order to find the best parameter for the inverse of regularization strength of the algorithm. The parameter is chosen from the set $\{\frac{1}{64}, \frac{1}{32}, \frac{1}{16}, \frac{1}{8}, \frac{1}{4}, \frac{1}{2}, 1, 2, 4, 8 , 16, 32 , 64\}$. The parameter with the highest accuracy on the validation set is chosen for the final model. The accuracy of the model is then evaluated on the test set.

\begin{table}[htbp]
    \centering
    \begin{tabular}{lll}
    \toprule
    Experiment & A. Id. ($n=1000$) & A. Id. ($n=50$) \\
    \midrule
    Size of Data (train/val/test) & 17306:962:961 & 1290:72:72 \\
    Accuracy (train/val/test) & 26.83\%/23.91\%/20.19\% &  55.50\%/51.39\%/56.94\% \\
    Precision & 0.229/0.159/0.132  & 0.528/0.453/0.408 \\
    Recall & 0.249/0.19/0.164 & 0.539/0.414/0.529 \\
    F1-Score & 0.216/0.158/0.134 & 0.518/0.414/0.431 \\
    \bottomrule
    \end{tabular}
         
    \caption{Performance results for Authorship Identification ($n=1000$ and $n=50$).} 
    \label{tab:authorship_identification}
\end{table}

\subsection*{Results}
Table \ref{tab:authorship_identification} details the results of the case study for both the 50 authors case and the 1000 authors case. Accuracy, precision, recall and F1 scores are reported. The model is able to distinguish between authors with an accuracy of 55.50\% for the 50 authors case, and 26.83\% for the 1000 authors case. The results also show that the model is not overfitting, as the accuracy on the test set is similar to the accuracy on the training set. Although not clearly visible, Figure \ref{fig:cmatrix_authorship_identification} shows the confusion matrix for the authorship identification experiment. 

Unfortunately, the model struggles to achieve high accuracy for a huge number of authors, in our case the total number being a thousand. However, if we consider a baseline of a simple coin flipping, that is, a model that will randomly assign a label from the available labels (authors) to each work with a probability of $\frac{1}{1000}$ (for the bigger classification case), we can see that the model does perform relatively well in distinguishing some features that are similar between authors. The confusion matrix potentially indicates similar features between authors, a characteristic that may be a topic for further research. Note the highest number of correct predictions (6 and 6 out of a maximum of 30) respectively in the middle of the matrix in \ref{cn_50} and towards the tail end of the diagonal of \ref{cn_1000} (around (855,855)). The other values are relatively low, with the highest number of incorrect predictions being 3 and 3, respectively. 

The results are, in fact, in line with the results of \cite{qian_deep_nodate}, who report an accuracy of 69.1\% on the Reuters 50\_50 (C50) dataset and 89.2 \% on the Gutenberg dataset (for a maximum of 50 authors and 45000 paragraphs of text from their works). The authors used sentence- and article-level GRUs and an article-level LSTM neural network to achieve these results. The authors also provided a baseline accuracy of 12.24\% via Gradient Boosting Classifier with 3 features, those being average word length, average sentence length, and Hapax Legomenon ratio (fraction of unique words), 2 of which we used. Given this baseline and the trivial computational complexity of our experiments, we have a reason to believe that we surpass the baseline and the researched features do enable stronger distinction between authors in this classification task.

Potential areas for improvement include the use of more sophisticated features, such as the ones described in \ref{heavyweight_metrics}, as well as the use of more sophisticated models for multiclass classification, such as SVMs, neural networks, or Naive Bayes classifiers. We then move on to the next case study, which is the identification of machine-generated text.

\begin{figure}[htbp]
    \begin{subfigure}[t]{0.5\textwidth}
        \includegraphics[height=2.5in]{../src/plots/authorship_identification/aid_50.png}
        \caption{$n=50$}\label{cn_50}
    \end{subfigure}
    \begin{subfigure}[t]{0.5\textwidth}
        \includegraphics[height=2.5in]{../src/plots/authorship_identification/aid_1000.png}
        \caption{$n=1000$}\label{cn_1000}
    
    \end{subfigure}
    \caption{Confusion Matrices for Authorship Identification. The rows represent the true labels, while the columns represent the predicted labels.}
    \label{fig:cmatrix_authorship_identification}

\end{figure}


\subsection{Machine-Generated Text Identification}
Having tested our hypothesis that the features are able to distinguish between different classes of documents in section \ref{sec:document_class_identification}, and then identified that the features are able to distinguish between different authors in section \ref{sec:authorship_identification}, we now move on to the ambitious goal of identifying machine-generated text. LLMs have seen explosive growth in the past few years, and generating human-like text, both grammatically and (somewhat) logically-sound, is now far from a distant dream. However, the ability to generate text indistinguishable from human-written text has raised concerns about the potential for misuse of such models. Particular issues may arise for example with academic grading, as AI writing tools become more and more prevalent. Fields other than academia may also be affected, such as journalism, where AI writing tools may be used to lazily generate non-proofread articles with the potential to spread misinformation. Of course, the potential for misuse is not limited to the above examples, and there are plenty of uses that malicious agents can come up with, either for personal gain or for the sake of spreading chaos.

Now then, we arrive back at the essence of the problem we are trying to solve: ``What defines creativity? What defines human creativity?'' The answer to this question is not simple, and it is not the goal of this thesis to answer it in full. However, we can attempt to answer a more specific question: ``Can a machine distinguish between human-written text and machine-generated text?''

\subsection*{Setup}
We make use of the WebText dataset \citep{radford2019_gpt2}, which is a large dataset of text scraped from the internet and used to train the influential GPT-2 (\acrlong{gpt}), a LLM developed by OpenAI. Texts generated by GPT-2 themselves serve as the basis for the ``machine-generated'' classification labels and are labelled as \acrfull{mgt}. Samples from the WebText dataset used to train GPT-2 form the basis for the ``human-written'' classification labels. 20,000 samples are randomly drawn from the set of machine-generated texts, and 20,000 --- from the set of human texts.

The dataset is split into a training, validation, and test set, with a ratio of 80:10:10. The training set is used to train a logistic regression with L2 penalty, which is then used to predict the class of the documents in the test set. As an intermediary step, we run a grid search with the training dataset and the validation dataset in order to find the best parameter for the inverse of regularization strength of the algorithm. The parameter is chosen from the set $\{\frac{1}{64}, \frac{1}{32}, \frac{1}{16}, \frac{1}{8}, \frac{1}{4}, \frac{1}{2}, 1, 2, 4, 8 , 16, 32 , 64\}$. The parameter with the highest accuracy on the validation set is chosen for the final model. The accuracy of the model is then evaluated on the test set. 

\subsection*{Results}
The model reports an accuracy of 69.7\% on the training set, 70.0\% on the validation set, and 70.0\% on the test set. The results show that the model is not overfitting, as the accuracy on the test set is similar to the accuracy on the training set. Figure \ref{fig:cmatrix_mgt_detection} shows the confusion matrix for the machine-generated text detection experiment. As seen, the document is able to distinguish between the classes with a passable accuracy, precision and recall. 


\begin{table}[htbp]
    \centering
    \caption{Performance results for MGT Detection}
    \label{tab:mgt_detection}
    \begin{tabular}{llll}
    \toprule
     & \multicolumn{3}{c}{Split} \\
     \cline{2-4}
     & Train & Val & Test \\
    \midrule 
    Size of Data & 32000 & 4000 & 4000 \\
    Accuracy & 0.697 & 0.700 & 0.700 \\
    Precision & 0.698 & 0.702 & 0.700 \\
    Recall & 0.697 & 0.700 & 0.700 \\
    F1-Score & 0.697 & 0.700 & 0.700 \\
    \bottomrule
    \end{tabular}
\end{table}

\begin{figure}[htbp]
    \centering
    \includegraphics[width=0.5\textwidth]{../src/plots/mgt_detection/cmatrix_xl.png} 
    \caption{Confusion Matrix for MGT Detection. The rows represent the true labels, while the columns represent the predicted labels.}
    \label{fig:cmatrix_mgt_detection}
\end{figure}

\subsection*{Discussion}
The results displayed by the model show some promise, especially given the trivial nature of the features we studied. Accuracy if higher, could be used to detect machine-generated text with a high degree of certainty. However, the results are not as high as we would like them to be, and there is a lot of room for improvement. As discussed in prior points, potential areas for improvement include the use of more sophisticated features, such as the ones described in \ref{heavyweight_metrics}, as well as the use of more sophisticated models for multiclass classification, such as SVMs, neural networks, or Naive Bayes classifiers.

The authors of GPT-2 provide a baseline of their own \footnote{\url{https://github.com/openai/gpt-2-output-dataset/blob/master/detection.md}}, citing 74.31\% accuracy for the temperature 1-sampled output of the XL version of the GPT-2 model (1582 billion
parameters), and 92.62\% for the K40-sampled output of the same model. Again, our results only serve as a demonstration of the accuracy of the \textsc{Mad Hatter} package, rather than a definitive leap in the field of machine-generated text detection. We do look forward, however, to more expansive testing of the package, as well as the implementation of more sophisticated features and models in the future.

\subsection{Summary}
In this section, we have implemented three experiments to evaluate the performance of \textsc{Mad Hatter}. We started by evaluating the performance of \textsc{Mad Hatter} in identifying the class of a document, thus demonstrating that the features we implement are well-defined and enable differentiating between different types of writing. We then moved on to evaluating how well the algorithm can differentiate between different writing styles, a task also known as authorship identification. Finally, we followed the logical progression of authorship identification to address a topic that has been gaining traction in recent years, that of machine-generated text detection. This may have further applications in the future, as the field of natural language generation has been steadily growing in the past few years, with the advent of \acrshort{llm}s such as GPT-2 \citep{radford2019_gpt2} and GPT-3 \citep{brown_gpt3_2020}. 

\chapter{Discussion and Conclusion}
\label{chap:discussion}

In the following chapter, we discuss the results of the experiments conducted in the previous chapter. We also discuss the limitations of the system and the results, as well as the future work that can be done to improve the system. Finally, we conclude with a summary of the contributions of this work. 

\section{Formal Evaluation}
We introduced \textsc{Mad Hatter}, a text processing and a linguistic benchmarking tool, as well as provided a set of benchmarks for evaluating the creativity of text. We have also tested our benchmarks on a set of texts, and provided a set of experiments that can be used to evaluate the system. 

\subsection{Evaluation of the system}
The system for benchmarking text we implemented is a useful tool for text analysis, that fills in a niche in the Python ecosystem for textual analysis that has not been filled before -- that of a tool for benchmarking text. As we mention, the system is also a strong tool for text analysis that can be used for a variety of baseline NLP tasks, such as text tokenization, part-of-speech tagging, and word frequency analysis.

\textsc{Mad Hatter} occupies a similar category of text analysis tools such as the Natural Language Toolkit (NLTK) \citep{nltk_citation}, the TextBlob library \citep{textblob}, and the spaCy library \citep{spacy2}. However, \textsc{Mad Hatter} is the only tool that provides a set of benchmarks for evaluating the creativity of text and handy operations to aid with visualization. 

Similarly to the tools above, Mad Hatter is extensible and can be used to build more complex tools for text analysis. Because \textsc{Mad Hatter} has been built on top of the NLTK library, it can be extended with any of the tools provided by NLTK in the future, as well. Yet, the implemented methods are also framework-agnostic, as the main functions have been built on top of pure Python, and would require minimal configuration to be borrowed by other frameworks such as \texttt{SpaCy}, \texttt{gensim}, and others. Furthermore, as a significant portion of the project was spent on researching methods for evaluating creativity in text, and the documentation in the source code is exhaustive, the project is easily portable in other, more performant programming languages such as C++ or Rust. We hope that this project can be used as a starting point for future research in the field.


\subsection{Evaluation of the benchmark}

We carried out an evaluation on the majority of the metrics implemented in the benchmark, for the purposes of determining the validity of the metrics. We examined the metrics' effectiveness in determining the source of the text -- their genre, e.g. article, short story, or a legal document. Then, within the same genre, we followed by examining the metrics' effectiveness in determining the author of the text. Finally, we examined the metrics' effectiveness in determining whether a given text has been written by a human, or a machine.

All metrics performed well \textbf{in providing distinctive features for the given texts}, and complemented each other to a high degree in the selected tasks. In terms of accuracy, the accomplished results fare well against the performance of alternative algorithms implemented for the same tasks and the same datasets. Furthermore, if we compare the methods providing said alternative algorithms, we can see that the metrics which underpin our data for the machine learning pipelines are much more lightweight and computationally efficient, and therefore have potential to be used at a large scale.

The fact that our metrics can uniquely identify authors with a relatively high accuracy given the trivial computational load of the lightweight metrics, suggests that improvements and additions to metrics evaluating the text in terms of its structure,  such as the ones we proposed in Section \ref{sec:llm_metrics}, can be used to further improve the accuracy of the metrics and yield more accurate results in the difficult task of authorship identification. \textbf{We conclude that creativity boils down to the expression of an author's unique character through their work. If our algorithms were able to capture that, then we can make the claim that our metrics are a good representation of creativity.}


% \subsection{Accomplished goals}


\subsection{Limitations}
We were limited in terms of the amount of time we could spend on the project, and the amount of resources we could use. Subsequently, that limited the amount and scale of experiments we could run, and the amount of data we could use. In terms of design, we have gained a valuable experience in working with NLP systems, and, if given the chance, would like to design the system in a more modular way, or such that it completely abstracts away the underlying NLP framework. We are fascinated with what we could do if we had access to a high-performance cluster, and would like to explore the possibilities of running the benchmarks on a larger scale. 

These limitations, however, leave room for future work, which we discuss in a following section.

\section{Contributions}

As discussed in the previous section, we have made a number of contributions to the field of computational creativity. We have implemented a system for benchmarking text, and a set of benchmarks for evaluating the creativity of text. We have also provided a set of experiments that can be used to evaluate the system. These are all non-negligible, as most of the work in the niche of computational evaluation of text has been done in other topics, such as sentiment analysis, or text classification -- both of which have major importance in business applications, of course. However, the field of computational creativity is still in its infancy, and we hope that our work can be used as a starting point for future research in the field, and potential applications in the business sphere.

\section{Future Work}

We have identified a number of areas for future work. Firstly, improvement of the existing system is a priority. We would like to improve the system in terms of its design, and make it more modular, and more extensible. We would also like to improve the system in terms of its performance and accuracy. We would like to explore the possibility of running the benchmarks on a larger scale, and on a high-performance cluster. We would also like to explore the possibility of running the benchmarks on a larger dataset, and on a more diverse dataset. 


Finally, we would like to explore the possibility of using the benchmarks for evaluating the creativity of text generated by LLMs, in light of recent advances in other ``creative'' AI, such as text-to-image generation and synthesis AI like Midjourney, Stable Diffusion \citep{stable_diffusion_rombach2022highresolution}, and DALL-E \citep{dall-e_ramesh2021zeroshot}.
% \include{sections/conclusion}

\appendix
% \include{proof}

\chapter{User Manual}
The following user manual lists and explains the features of the Mad Hatter package. It also provides a guide on how to install and use the package, as well as how to use the command-line interface. The guide assumes that the user \textbf{has installed Python 3.7+}, and has access to a terminal.

\section{Installation}

Run the following command to install the package and its dependencies:

\begin{lstlisting}[language=bash]
    pip install madhatter
\end{lstlisting}

We highly recommend also running NLTK's downloader module in order to have access to all of the features that Mad Hatter provides. To do so, simply run the following command:

\begin{lstlisting}[language=bash]
    python -m nltk.downloader all
\end{lstlisting}


\subsection{Usage}

The package provides high-level abstractions for text analysis that can be used with any text. The following example shows how to use the package to analyse a simple text file within Python:

\begin{lstlisting}[language=python, breaklines]
from madhatter.benchmark import CreativityBenchmark

text = "The quick brown fox jumped over the lazy dog."
bench = CreativityBenchmark(text)

bench.report()
>>> BookReport(title='unknown', nwords=10, mean_wl=3.7, mean_sl=45.0, mean_tokenspersent=10.0, prop_contentwords=0.1, mean_conc=4.0633333333333335, mean_img=5.359999999999999, mean_freq=-1.6792249660842167, prop_pos={'ADJ': 0.2, 'NOUN': 0.3, 'VERB': 0.1}, surprisal=None, predictability=None)
\end{lstlisting}


\subsection{Command Line Interface}
Mad Hatter is also available as a CLI tool. Simply provide a path to a text file to the CLI, and it will generate a report for that text. Table \ref{tab:cli_options} lists the available options to the CLI. The CLI must be supplied a filename or a path to the file to be analysed. 
The following example shows how to use the CLI to generate a report for an arbitrary text file:

\begin{lstlisting}[language=bash, breaklines]
> python -m madhatter text.txt -p -m 100 -c 20 -t "My Text"

BookReport(title='My Text', nwords=100, mean_wl=3.7, mean_sl=45.0, mean_tokenspersent=10.0, prop_contentwords=0.1, mean_conc=4.0633333333333335, mean_img=5.359999999999999, mean_freq=-1.6792249660842167, prop_pos={'ADJ': 0.2, 'NOUN': 0.3, 'VERB': 0.1}, surprisal=None, predictability=None)
\end{lstlisting}

\begin{table}[h!]
    \centering
    \begin{tabular}{lllp{0.6\textwidth}}
        \toprule
        Short & Long Form & Default & Description \\
        \midrule
        \texttt{-h} & \verb|--help| & & Show a help message and exit the program.\\
        \texttt{-p} & \verb|--postag| & & Whether to return a POS tag distribution over the whole text. The option is a flag, so it only needs to be added. \\
        \texttt{-u} & \verb|--usellm| & & Whether to run GPU-intensive LLMs for additional characteristics. The option is a flag, so it only needs to be added. \\
        \texttt{-m} & \verb|--maxtokens| & -1 & Maximum number of predicted words for the heavyweight metrics. Count starts from the beginning of text, -1 to read until the end. \\
        \texttt{-c} & \verb|--context| & 10 & Context length for sliding window predictions as part of heavyweight metrics. Longer context for better results, but may potentially result longer compute times.\\
        \texttt{-t} & \verb|--title| & 
         & Optional title to use for the report project. If not supplied, the name of the file supplied is used in the book report.\\
        \texttt{-d} & \verb|--tagset| & universal & Tagset to use. Default is the universal tagset.\\
        \bottomrule
      
    \end{tabular}
    \caption{Available options to the CLI.}\label{tab:cli_options}
\end{table}


\subsection{Advanced Usage}

Users are also able to use the package's lower-level functions to create their own custom analysis pipeline or to integrate with other NLP packages such as \href{https://github.com/explosion/spaCy}{SpaCy}.

\begin{lstlisting}[language=python, breaklines]
    from madhatter import metrics
    from madhatter import benchmark
    
    text = "The quick brown fox jumped over the lazy dog."
    bench = benchmark.CreativityBenchmark(text)
    
    bench.words
    >>> ['The', 'quick', 'brown', 'fox', 'jumped', 'over', 'the', 'lazy', 'dog', '.']
    
    metrics.imageability(bench.words)
    >>> [1.41, 2.45, 3.14, 4.2, 3.4, 3.65, 1.41, 2.42, 4.1, 0.0]
\end{lstlisting}

\chapter{Maintenance Manual}

This should be used to describe the details of your implementation. It should be usable by people wanting to install the program, modify the program, extend the program, or trace bugs in its execution. This is an important part of the documentation, and you should ensure that you include details such as:
• instructions on how to install the system
• instructions on how to compile/build the system
• hardware/software dependencies, including libraries and other packages
8
• Organisation of system files, including directory structures, location of files within directories, details of any temporary files
• space and memory requirements
• list of source code files, with a summary of their role
• crucial constants, and their location in the code
• the main classes, procedures, methods or data structures
• file pathnames, particularly for accessing files of data values • directions for future improvements
• bug reports
Again, the Maintenance Manual must be included as an appendix to the main report and will therefore be marked as part of the disserta- tion. (this is in addition to a copy of the maintenance manual with the code tar file)
Here is an example template project report document, and here is a tarfile of the corresponding LaTeX source (on Unix, just type “make”).

\bibliography{bibliography}

\end{document}
