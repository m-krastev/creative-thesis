\chapter{Related Work}
\label{chap:background}
The field of creativity research has been studied for decades, and there are many different approaches to the problem. In this section, we will discuss the different approaches to creativity research, and how they relate to our work. We will also discuss the different approaches to creativity in the context of natural language processing, and how they relate to our work.
% Include background, what previous authors have done in a mostly neutral style. Optimally expecting ~1000-2000 words and 30-40 references mentioned.

\section{Challenge Landscape}

\subsection{Part of Speech Tagging}

\subsection{Word Sense Disambiguation}

\subsection{Sentiment Analysis}

% Maybe not discuss this in general
\subsection{Phonetic Analysis}

\subsection{Natural Language Generation}
\begin{itemize}
    \item top KP sampling
    \item challenges 
\end{itemize}


\section{Creative Measures}
\begin{itemize}
    \item Burstiness of verbs and derived nouns: Patterns of language are sometimes `bursty' \cite{pierrehumbert_burstiness_2012}. This paper presents an analysis of text patterns for domain X. measures include XYZ...
    \item 
\end{itemize}

\section{Tools}
