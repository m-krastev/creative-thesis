\chapter{Related Work}
\label{chap:background}
The field of creativity research has been studied for decades, and there are many different approaches to the problem. In this section, we will discuss the different approaches to creativity research, and how they relate to our work. We will also discuss the different approaches to creativity in the context of natural language processing, and how they relate to our work.
% Include background, what previous authors have done in a mostly neutral style. Optimally expecting ~1000-2000 words and 30-40 references mentioned.

\section{Challenge Landscape}
\label{sec:challenge_landscape}
Most of the work going on in our minds as we read a given text is unconscious. We automatically parse characters and tokenize the text into words, sentences, paragraphs, and so on. At the same time, we apply tagging to understand actors (subjects and objects), actions (verbs), setting and situation (adverb), and also try to understand the sense a given word is used in, and then we transform the connotations or the sense of words inside our own little mind representation of the words. We also apply sentiment analysis to understand the emotional state of the text, and we apply phonetic analysis to understand the pronunciation of the words. 
Firstly, most of the work that goes on in our minds as we read a given text 

The field of creativity has been broadly studied, initially by psychologists, and more recently by computer scientists. Intutively, the nature of creativity is a subjective matter, and therefore, it is difficult to define. However, there are some commonalities that can be found in creative works. For example, creativity is often associated with novelty, and is often associated with the ability to generate new ideas. \cite{franceschelli_deepcreativity_2022}, for example, determine the three factors of creativity as value, novelty, and surprise, and then explore machine learning approaches to measuring creativity. We agree with them, but decide not to limit ourselves to just these three. However, their contributions provide insights we can use in our work. \mk{did we use it - if we did not, delete this sentence} 

\subsection{Part of Speech Tagging}

\subsection{Word Sense Disambiguation}

\subsection{Sentiment Analysis}

% Maybe not discuss this in general
\subsection{Phonetic Analysis}

\subsection{Natural Language Generation}
\begin{itemize}
    \item top KP sampling
    \item challenges 
\end{itemize}


\section{Creative Measures}
\begin{itemize}
    \item Burstiness of verbs and derived nouns: Patterns of language are sometimes `bursty' \cite{pierrehumbert_burstiness_2012}. This paper presents an analysis of text patterns for domain X. measures include XYZ...
    \item 
\end{itemize}

\section{Tools}
