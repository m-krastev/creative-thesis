\chapter{Introduction}
\label{chap:intro}

\section{Inspiration}
At the heart of the field of artificial intelligence is the concept of reproducing aspects defining human intelligence through rigorous examination and replication of the mechanisms that drive progress. This is a uniquely multi-faceted problem with a multitude of approaches, each tailored to a very specific aspect or manifestations of intelligence. Naturally, intelligence assumes following a logical pathway to arrive at sensible conclusions that interact with the real world beneficially. This can be viewed through the lens of methodical, defined process that always follows a certain formula. An organism, assumed to be intelligent, might always follow such formulaic actions given a set of prerequisite conditions to accomplish a defined goal. Certain schools of thought theorise that there is such an order in every little action, and such thread of logicality interweaves every law of nature, known or not. Then, follows the question, can we recognize and define such a thread for the ambitious field of creativity? We seek not to properly define or constrain the subject of creativity, rather, we explore markers of what could only be a subset of the very broad field of human creativity.

The subject of the present document is exclusively the study of linguistic creativity. Henceforth, we seek to: confirm prior results of the research of psycholinguistics, affirm that hypotheses and conclusions drawn from them correlate highly with certain manifestations or aspects of creativity, and make firm the subject of creativity, that is, provide tools that may be used for exploration and analysis of specific creative features found in text.

%% following section copied from project plan
\section{Motivation}

Large language models (LLMs) are probabilistic models of language widely used for most tasks in the field of Natural Language Processing (NLP), ranging from machine translation, text classification, sentiment analysis, auto-completion, error correction, or even simple dialogue communication. However, because, fundamentally, LLMs operate on probabilities, a lot of the applications utilizing them tend to struggle with generating logically coherent novel sequences. 

Large language models are usually trained on enormous language corpora, mined from books and articles \citep{gutenberg_dataset}, social media posts \citep{broad_twitter}, or otherwise internet crawls \citep{thepile_dataset}. Therefore, the underlying assumption would be that, in their attempt to replicate language, they could find some success at least in terms of generating basic structure in creative fields of work such as writing code \citep{codex_2021_copilot} or screenplays \citep{mirowski_co-writing_2022}, among others. 

While LLMs display convincingly human-like text in such areas, we instead intend to examine their ability to produce specifically \emph{creative} text. By its very nature, creativity tends to be a subjective matter, thus, we aim to identify specific aspects of creativity that can be more commonly found within creative text, such as poetry or short stories, as opposed to other less creatively-oriented genres, e.g. news articles or user manuals. Having identified specific linguistic markers within human-written creative texts, we can then begin to evaluate the extent to which LLMs exhibit these properties in generated text. This in turn has the potential to inform techniques to adapt existing text-generation models for applications involving creative language. Therefore, we seek to compile and produce a software package that can efficiently and accurately represent and evaluate linguistic creativity. 

\section{Objectives}

We set out to investigate specific markers defining or correlating with conventional creativity in language. As some aspects of creativity have been investigated by research disciplines such as the field of psycholinguistics, we seek to filter and compile a set of measures that have been shown to correlate more highly with creative texts in the literature. 
Following that, we aim to evaluate the extent to which current LLMs can provide insights into capturing creative elements or patterns within writing. Finally, if time allows, we may use these findings to explore how natural language generation can be adapted to produce texts exhibiting more human-like levels of the creative attributes we have discovered. These can be summarised by the following three questions:

\begin{itemize}
    \item Can psycho-linguistically motivated measures (that is, the explored metrics) successfully characterise creative properties in language?
    \item Can a machine learning approach be adapted for evaluating creativity in natural language?
    \item Can we influence a subset of current LLMs to exhibit more creative traits as defined by our creativity metric via some conventional approaches such as hyperparameter tuning or varying decoding strategies? 
\end{itemize}

For the first two, we develop a suite of benchmarks we shall distribute and report the results of. 
For the third question, we plan to train a model maximising performance on the developed benchmarks. We will then report our findings on the best-performing models and share the architecture.

\section{Contributions}

The contributions of this work are as follows:

\mk{<insert rolling tumbleweed>}