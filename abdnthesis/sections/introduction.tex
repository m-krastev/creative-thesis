\chapter{Introduction}
\label{chap:intro}

\section{Inspiration}
At the heart of the field of artificial intelligence is the concept of reproducing aspects defining human intelligence through rigourous examination and replication of the mechanisms that drive progress. This is a uniquely multi-faceted problem with a multitude of approaches, each tailored to a very specific aspect or manifestations of intelligence. Naturally, intelligence assumes following a logical pathway to arrive at sensible conclusions that interact with the real world in a beneficial way. This can be viewed through the lens of methodical, defined process that always follows a certain formula. An organism, assumed to be intelligent, might always follow such formulaic actions given a set of prerequisite conditions to accomplish a defined goal. Certain schools of thought theorise that there is such an order in every little action, and such thread of logicality interweaves every law of nature, known or not. Then, follows the question, can we recognise and define such a thread for the ambitious field of creativity? We seek not to properly define or constrain the subject of creativity, rather, we explore markers of what could only be a subset of the very broad field of human creativity.

The subject of the present document is exclusively the study of linguistic creativity. Henceforth, we seek to: confirm prior results of the research of psycholinguistics, affirm that hypotheses and conclusions drawn from them correlate highly with certain manifestations of creativity, and make firm the subject matter of creativity, that is, provide tools that may be used for exploration and analysis of specific creative features found in text.