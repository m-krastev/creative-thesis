\chapter{Maintenance Manual}

This should be used to describe the details of your implementation. It should be usable by people wanting to install the program, modify the program, extend the program, or trace bugs in its execution. This is an important part of the documentation, and you should ensure that you include details such as:
• instructions on how to install the system
• instructions on how to compile/build the system
• hardware/software dependencies, including libraries and other packages
8
• Organisation of system files, including directory structures, location of files within directories, details of any temporary files
• space and memory requirements
• list of source code files, with a summary of their role
• crucial constants, and their location in the code
• the main classes, procedures, methods or data structures
• file pathnames, particularly for accessing files of data values • directions for future improvements
• bug reports
Again, the Maintenance Manual must be included as an appendix to the main report and will therefore be marked as part of the disserta- tion. (this is in addition to a copy of the maintenance manual with the code tar file)
Here is an example template project report document, and here is a tarfile of the corresponding LaTeX source (on Unix, just type “make”).