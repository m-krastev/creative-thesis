\chapter{Methodology}
\label{chap:methods}

In this chapter, we explore the different datasets to be used, the methods for evaluating creativity, and the algorithms for creativity evaluation. We will furthermore discuss the strengths and limitations of the proposed methods and algorithms, including time complexity, memory constraints, and the accuracy of the results. We take an informed approach to the selection of the datasets and the methods for evaluating creativity, and discuss the reasons for our choices in detail, and support them with relevant literature as explored by other researchers in the field.

\section{Datasets}
\label{sec:datasets}
Datasets are vital for the success of any given project in the field of machine learning, and even more so when concerning linguistics. As evidenced by \cite{torralba_unbiased_2011}, the quality of the data used for training a model has a direct impact on the quality of the results. A model trained on a specific dataset, e.g. a corpus of law documents, can be expected to perform poorly on a dataset of medical documents, as the two domains are inherently different. Thus, we take particular care in planning and selecting the datasets we use. We also consider ease of use and access, as some datasets may require additional processing, others are subject to availability issues (e.g. paid datasets and corpora), and some may be too large to be used in a reasonable amount of time.
In this section, we will explore the datasets used in this project, and discuss their strengths and limitations.
\subsection{Brown Corpus}
The Brown Corpus \citep*{francis1979brown} is a widely used corpus in the field of computational linguistics, noted for the small variety of genres of literature it contains. The Corpus itself is founded on a compilation of American English literature from the year 1961. It is also small in terms of size, totalling around one million words, at least compared to modern corpora, which we also explore later on. The corpus also suffers from the issue of recency, as the works and language may be outdated for modern speakers of English.

Of interest is the fact that the corpus has been manually tagged for parts of speech, a process that tends to be error-prone. %citation good
As we will see later on, this fact has implications in terms of the supervised learning algorithms we implement for creativity evaluation. Still, we opt to utilize it primarily for prototyping purposes and drawing preliminary conclusions about the effectiveness of the implemented algorithms, rather than in-depth analysis and publication of results.

\subsection{Project Gutenberg}
Project Gutenberg\footnote[1]{\url{https://www.gutenberg.org/}} is a large collection of more than 50,000 works available in the public domain. The collection contains literature from various years and various genres and thus is suitable for training and evaluation of the developed benchmarks in the context of creativity study. 

As the Project does not offer an easy to process copy of its collection, we turn to the work of \cite{DBLP:journals/corr/abs-1812-08092}. The team developed a catalogue for on-demand download of the entire set of books available on the Project Gutenberg website, intended for use in the study of computational linguistics. The tool avoids the overhead of writing a web-scraper or a manual parser for the downloadable collections of Project Gutenberg books made available by third parties, as well as enables easy synchronization of newly released literature. Instead, we are only required to develop a simple pipeline for the data to be fed into the utilized systems. 

\subsection{Hierarchical Neural Story Generation}
In their work, \cite{fan_hierarchical_2018} trained a language model for text generation tasks on a dataset comprised of short stories submitted by multiple users given a particular premise (a prompt or a theme) by another user. \mk{Give an example for how one such short story would look like.} The dataset in question is technically referred to a series of posts and comments (threads) to them on the popular social media platform \textsc{Reddit}, and more tightly, the \textit{subreddit} forum \textsc{r/WritingPrompts}. The authors of the work \cite{fan_hierarchical_2018} have made the dataset available for public use, and we have used it for the purpose of evaluating the performance of our creativity benchmarks. As described by the authors on their GitHub page\footnote{\url{https://github.com/facebookresearch/fairseq/blob/main/examples/stories/README.md}}, the paper models the first 1000 tokens (words) of each story.

\mk{How do we use this dataset? You should describe the process of how we use it. }

\subsection*{Discarded Datasets and Corpora}
Some datasets were considered, however, discarded due to: not being deemed applicable for the context of the application; general lack of availability of the dataset in a form that is easily accessible for our purposes; simply being infeasible to use due to the size of the dataset and the hardware constraints imposed on the project; or other reasons of similar nature.

\subsubsection*{The COCA}
The Corpus of Contemporary American English (COCA)\footnote[2]{\url{https://www.english-corpora.org/coca/}} is a large corpus of American English, containing nearly 1 billion words of text from contemporary sources. It is a collection of texts from a variety of genres, including fiction, non-fiction, and academic writing. The corpus offers a variety of tools for analysis of the data, including a concordance tool, a word frequency list, and a collocation finder. Naturally, many of those tools could be used in the field of statistical creativity analysis that we explore.

The corpus does offer limited access to the full API, as well as free samples of the data, however, the full corpus is not available for free, and the cost of acquiring it is prohibitive for the limitations set forward by the project. Nevertheless, the corpus is a valuable resource for the field of computational linguistics, and we would like to explore it further given less constraints.
\section{WordNet}
WordNet\citep{wordnet1998fellbaum} is a lexical database of semantic relations between words that links words into semantic relations including synonyms, hyponyms, and meronyms. The synonyms are grouped into synsets (sets of synonyms) with short definitions and usage examples. It can thus be seen as a combination and extension of a dictionary and thesaurus \citep{enwiki:1143619785}. 

For our specific use cases, we have identified it as a valuable resource in terms of relational representation of words in semantic space. In the given context, this enables us to traverse a semantic graph for synonyms and related words for the goal of enriching potential similarity between the set of creative parts of speech (i.e., nouns, adjectives, adverbs), which we narrow down our scope to in particular. 
% this sounds a bit weird
% insert potential uses
\subsection{Numerical representations of semantic tokens}
\mk{Potentially move this section to background work}
The idea of representing words or lexical tokens as numerical vectors (or even scalars) is hardly new. 
For example the SimLex-999 dataset \citep*{hill-etal-2015-simlex} gives values on a scale from 0 to 10, like the examples below, which range from near-synonyms (vanish, disappear) to pairs that scarcely seem to have anything in common (hole, agreement):

\begin{table}[htbp]
    \centering
        \begin{tabular}{llc}
            \toprule
            word1 & word2 & score \\
            \midrule
            vanish & disappear & 9.8 \\
            hole & agreement & 1.2 \\
            \bottomrule
       \end{tabular}
    \caption{Example Simlex-999 pairs}
    \label{simlex2pairs}
\end{table}


Early work on affective meaning by \cite{osgood1957measurement} found that words varied along three important dimensions of affective meaning:
\begin{itemize}
    \item valence: the pleasantness of the stimulus
    \item arousal: the intensity of emotion provoked by the stimulus
    \item dominance: the degree of control exerted by the stimulus
\end{itemize}

\cite{osgood1957measurement} noticed that in using these 3 numbers to represent the meaning of a word, the model was representing each word as a point in a three-dimensional space, a vector whose three dimensions corresponded to the word’s rating on the three scales. This revolutionary idea that word meaning could be represented as a point in space (e.g., that part of the meaning of heartbreak can be represented as the point $[2.45,5.65,3.58]$) was the first expression of the vector semantics models that we introduce next. \mk{**You can paraphrase this**}

\subsection*{Word2Vec}

\citet*{mikolov_word2vec_2013} show in their work that words may be represented as dense vectors in $N$-dimensional space, and we can perform mathematical operations on them that may yield effective results in terms of word representation. 

\subsection*{Measuring distance in vector representations of semantic tokens}
Intuition tells us that the dot product of vectors in $N$-dimensional space will grow when the set of vectors has similar values and decrease when the values are not similar. Thus, we can then construct the following metric for semantic similarity between vector representations of words:
$$ D(v,w) = v \times w = \sum_{i=1}^{N} v_i w_i = v_1 w_1 + v_2 w_2 + \dots + v_N w_N $$ 

The current metric, however, suffers from the problem that vectors of higher dimensions will inevitably be larger than vectors with lower dimensions. Furthermore, embedding vectors for words that occur frequently in text, tend to have high values in more dimensions, that is, they correlate with more words. The proposed solution is to normalize using the \textbf{vector length} as defined:
$$ | v| = \sqrt{\sum_{i=1}^{N}v_i^2}$$

Therefore, we obtain the following:

$$ \text{Similarity} (v, w) = \frac{v \times w}{|v| |w|} = \frac{\sum_{i=1}^{N} v_i w_i}{\sqrt{\sum_{i=1}^{N}v_i^2} \sqrt{\sum_{i=1}^{N}w_i^2}}$$

This product turns out to be the same as the cosine of the angle between two vectors:

$$ \frac{a \times b}{|a| |b|} = \cos(\theta) $$

Therefore, we will call this metric the \textbf{cosine similarity} of two words. As mentioned, the similarity grows for vectors with similar features along the same dimensions. Note the boundaries of said cosine metric: we get $-1$ for vectors which are polar opposites, $0$ for orthogonal vectors, and $1$ for equivalent vectors. Of note is the fact that such learned vector embeddings only have values in the positive ranges, thus, it is impossible to have negative values for the cosine similarity (Similarity$(a,b) \in [0,1]$).

Contrary to it, we also identify the metric of \textbf{cosine distance} between two vectors, as one minus the similarity of the vectors, or:

$$ \text{Distance}(v,w) = 1 - \text{Similarity}(v,w) $$

The cosine distance may prove useful when dealing with minimisation problems as is often the case with machine learning.

\section{Metrics}
\subsection{Concreteness}
\label{concreteness}
Concreteness is the degree to which a word refers to a tangible object or a concrete idea. For example, the word \textit{apple} is concrete, while the word \textit{love} is abstract. \cite{brysbaert2014concreteness} provide a dataset of concreteness ratings for 40,000 English lemmas (English words and 2,896 two-word expressions (such as ``zebra crossing'' and ``zoom in''), obtained from over four thousand participants by means of a norming study using internet crowdsourcing for data collection). The dataset is based on the concreteness ratings of the four thousand participants, who rated the concreteness of 40,000 words on a scale from 1 to 5. The dataset also contains the number of participants who rated each word, and the standard deviation of the ratings. 
 The concreteness of a word is measured on a scale from 1 to 5, where 1 is the most abstract and 5 is the most concrete. The concreteness of a word is measured by asking people to rate the concreteness of a word on a scale from 1 to 5. The concreteness of a word is measured by asking people to rate the concreteness of a word on a scale from 1 to 5. The concreteness of a word is measured by asking people to rate the concreteness of a word on a scale from 1 to 5. The concreteness of a word is measured by asking people to rate the concreteness of a word on a scale from 1 to 5. The concreteness of a word is measured by asking people to rate the concreteness of a word on a scale from 1 to 5. The concreteness of a word is measured by asking people to rate the concreteness of a word on a scale from 1 to 5. The concreteness of a word is measured by asking people to rate the concreteness of a word on a scale from 1 to 5. The concreteness of a word is measured by asking people to rate the concreteness of a word on a scale from 1 to 5.